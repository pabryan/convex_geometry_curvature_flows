\chapter{The Flow}

\begin{itemize}
\item Introduce the flows
\item Diffeomorpism invariance
\item Examples
\item Motivation. For our applications most flows will be gradient flows?
\end{itemize}

\chapter{One Parameter Families and Variations of Geometry}

\begin{itemize}
\item Needed for linearisation: existence/uniqueness and stabiilty
\item Needed for evolution equations
\item write as tangential + normal part
\item Space-time approach
\end{itemize}

\chapter{Curvature Functions}

\begin{itemize}
\item Expressions in terms of power sums and elementary symmetric functions
\item Functions of \(\kappa_i\) vs functions of \(W\)
\item Naturality/Invariance under parallel transport/Vertical derivatives
\item Differentials
\item Anisotropic factors and support function dependence
\end{itemize}

\chapter{Evolution Equations}

\begin{itemize}
\item Express variations in terms of curvature functions
\item Space time approach
\end{itemize}

\chapter{Existence, Uniqueness and Regularity}

\begin{itemize}
\item Apply linearisation and PDE theory
\item de Turk Trick
\item Degenerate paraboic: geometric invariance handled by de Turk but e.g. GCF becomes degenerate near boundary of convex cone
\item smoothing estimates: Krylov-Safanov
\item used for long time existence and convergence
\item Guan estimate?
\item Pogorelov?
\item Brendle-Choi-Daskalopoulos?
\end{itemize}

\chapter{Preserving convexity}

\begin{itemize}
\item Preserving convexity, mean convexity etc.
\item Preserving cones
\end{itemize}

\chapter{Affine Differential Geometry}

\begin{itemize}
\item used for solitons and Harnack inequalities
\item surely has many applications for the convex geometry part!
\item affine normal flow
\item Perhaps this goes in the convex geometry section. For this section we don't need much of the theory and can defer the affine geometric interpretation to the convex geometry section (with referece of course!)
\end{itemize}

\chapter{Solitons and the Harnack Inequality}

\begin{itemize}
\item Gauss map param
\item reparam and BIS Harnack
\item Classification of solitons
\end{itemize}

\chapter{Curvature Bounds}

\begin{itemize}
\item in/out radius reminder
\item Tso's method
\item Displacement bounds
\item Harnack to get lower bounds
\end{itemize}

\chapter{Curvature balls}

\begin{itemize}
\item \(\alpha\) non-collapsing
\item in/out balls
\item Gives sharper control than in/out radius
\item Also sharper control than min/max principal curvatures
\end{itemize}

\chapter{Polar Bodies}

\begin{itemize}
\item Another way to get lower bounds
\item Particuarly nice given we want to do convex geometry later!
\end{itemize}

\chapter{Linear Stability and Ancient Solutions}

\begin{itemize}
\item Monotone quantities and classification
\item Do we have applications in convex geometry?
\item Classification tends to match nicely with convex/affine geometry in that we obtain spheres, ellipsiods
\end{itemize}

\chapter{Width Lemma}

\begin{itemize}
\item Ben's ``geometric lemma'' is another way to approach convergence which ties in nicely with convex geometry
\end{itemize}

\chapter{Normalised Flows and Convergence}

\begin{itemize}
\item long time existence
\item interpolation and smoothing for higher regularity and convergence
\end{itemize}