\section{The Flow}

\begin{itemize}
\item Introduce the flows
\item Diffeomorpism invariance
\end{itemize}

\section{One Parameter Families and Variations of Geometry}

\begin{itemize}
\item Needed for linearisation: existence/uniqueness and stabiilty
\item Needed for evolution equations
\item write as tangential + normal part which is
\end{itemize}

\section{Curvature Functions}

\begin{itemize}
\item Expressions in terms of power sums and elementary symmetric functions
\item Functions of \(\kappa_i\) vs functions of \(W\)
\item Naturality/Invariance under parallel transport/Vertical derivatives
\item Differentials
\item Anisotropic factors and support function dependence
\end{itemize}

\section{Evolution Equations}

\begin{itemize}
\item Express variations in terms of curvature functions
\end{itemize}

\section{Preserving convexity}

\begin{itemize}
\item Preserving convexity, mean convexity etc.
\item Preserving cones
\end{itemize}

\section{Existence and Uniqueness}

\begin{itemize}
\item Apply linearisation and PDE theory
\end{itemize}

\section{Regularity}

\begin{itemize}
\item smoothing estimates
\item used for long time existence and convergence
\item Guan estimate
\item Pogorelov
\item Brendle-Choi-Daskalopoulos
\end{itemize}

\section{Affine Differential Geometry}

\begin{itemize}
\item used for solitons and Harnack inequalities
\item surely has many applications for the convex geometry part!
\item affine normal flow
\end{itemize}

\section{Solitons and the Harnack Inequality}

\begin{itemize}
\item Gauss map param
\item reparam and BIS Harnack
\end{itemize}

\section{Curvature Bounds}

\begin{itemize}
\item in/out radius reminder
\item Tso's method
\item Displacement bounds
\item Harnack to get lower bounds
\end{itemize}

\section{Curvature balls}

\begin{itemize}
\item \(\alpha\) non-collapsing
\item in/out balls
\item Gives sharper control than in/out radius
\item Also sharper control than min/max principal curvatures
\end{itemize}

\section{Polar Bodies}

\begin{itemize}
\item Another way to get lower bounds
\item Particuarly nice given we want to do convex geometry later!
\end{itemize}

\section{Linear Stability and Ancient Solutions}

\begin{itemize}
\item This can have some nice connections with isoperimetric inequalities, classification and e.g. affine normal flow
\end{itemize}

\section{Width Lemma}

\begin{itemize}
\item Ben's ``geometric lemma'' is another way to approach convergence which ties in nicely with convex geometry
\end{itemize}

\section{Normalised Flows and Convergence}

\begin{itemize}
\item interpolation and smoothing for higher regularity
\end{itemize}