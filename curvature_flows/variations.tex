\section{One Parameter Families}

\subsection*{One Parameter Families of Maps}

\begin{defn}
A \emph{one-parameter family} of maps \(M \to N\) is a map
\[
F : M \times I \to N
\]
where \(I\) is an interval. We write \(F_t = F(\cdot, t)\). A one-parameter family of embeddings (immersions) is a one parameter family such that each \(F_t\) is an embedding (immersion).
\end{defn}

\begin{defn}
Let \(F_t\) be a one-parameter family of maps. The \emph{variation vector field},
\[
V_t = F_{\ast} \partial_t
\]
For each \(t\), \(V_t \in \Gamma(F_t^{\ast} TN)\). We also sometimes use the notation \(\dot{F} = \partial_t F = V\).
\end{defn}

\subsection*{Variations of One Parameter Families of Maps}

\begin{defn}
Let \(G : M \to N\). A \emph{variation} of \(G\) is a one-parameter family of maps \(F_t\), \(t \in (-\epsilon, \epsilon)\) such that \(F_0 = G\).
\end{defn}

In the case of variations, we are interested in the map \(G\) and not typically the full one-parameter family \(F_t\). In this case, the variation vector is \(V_0\).

\begin{lemma}[Existence of Variations]
Let \(W \in \Gamma(G^{\ast} N, M)\). Then there exists local variations \(F_t\) of \(G\) such that \(V_0 = \dot{F}_{t=0} = W\) locally. If \(M\) is compact, then there exists a global variation \(F_t\) with \(V_0 = \dot{F}_{t=0} = W\).
\end{lemma}

\begin{proof}
Certainly this works for immersion but perhaps it works in the generality given here?
\end{proof}

Let us now compute the variation of the differential \(dF_t\). The principle point to keep in mind is that \(dF_t \in \Gamma(T^{\ast} M \otimes F_t^{\ast} TN)\) which is a changing bundle with respect to \(t\) and we must account for this change. Another way to think of this is that the fibre of \(F_t^{\ast} TN\) over \(x\) is \((F_t^{\ast} TN)_x = T_{F_t(x)} N\). We must take into account the variation of the base point \(F_t(x)\). In other words, we can't directly compute the variation in the fibre since the fibre is itself varying and thus require a connection \(D\) on \(TN\) to ``connect'' the fibres.

\begin{prop}[Variation of the Differential]
The variation of \(dF_t\) is
\[
D_t dF_t = DV.
\]
\end{prop}

\begin{proof}
For a fixed tangent vector \(X \in TM\), since \([\partial_t, X] = 0 \Rightarrow [V, F_{\ast} X] = 0\) we compute
\[
\begin{split}
D_t dF_t (X) &= D_V (dF_t (X)) = D_V F_{\ast} X \\
&= D_{F_{\ast} X} V + [V, F_{\ast} X] \\
&= DV (F_{\ast} X).
\end{split}
\]
\end{proof}

\begin{rem}
It's convenient, partricularly when considering the variation of the connection and the curvature tensor, to have the variation of the the push forward of the Lie bracket:
\[
D_t F_{\ast} [X, Y] = D_t [F_{\ast} X, F_{\ast} Y] = DV(F_{\ast} [X, Y]).
\]
\end{rem}

Higher derivatives of \(F\) involve the variation of the connection and so we postpone the computation of these variations until we have computed the variation of connections.

\section{Variation of Geometry}

\subsection*{Variations of One Parameter Families of Tensor Fields}

For one-parameter families of tensor fields tangent to \(M\) - i.e. in \(T^p_q M\) we may compute the variation by differentiating in the fibre and thus require no connection. This is similar (but not identical) to the Lie derivative.

\begin{defn}
A one-parameter family of tensor fields of type \((p, q)\) is a map
Let \(T : M \times (-\epsilon, \epsilon) \to T^p_q M\) such that \(\pi^p_q \circ T(x, t) = x\) for all \(x\) where \(\pi^p_q : T^p_q M \to M\) is the bundle-projection. The variation of \(T\) is defined as the derivative in the fibre:
\[
\partial_t T (x, t) = \lim_{h\to 0} \frac{T(x, t + h) - T(x, t)}{h}.
\]
Equivalently,
\[
(\partial_t T_t) (X_1, \cdots, X_p, \alpha^1, \cdots, \alpha^q) = \partial_t \left[T_t (X_1, \cdots, X_p, \alpha^1, \cdots, \alpha^q)\right]
\]
\end{defn}

\begin{rem}
Taking partial derivatives, \(\partial_t\) commutes with taking traces. That is, given \(T_t \in \Gamma(T^p_q M)\) a one-parameter family of \((p,q)\) tensor fields and \(X \in TM\) we may take the trace
\[
\left(\Tr (T_t \otimes X)\right) (X_1, \cdots, X_{p-1}, \alpha^1, \cdots, \alpha^q) = T_t(X, X_1, \cdots, X_{p-1}, \alpha^1, \cdots, \alpha^q).
\]
Then
\[
\partial_t \left(\Tr (T_t \otimes X)\right) = \Tr \left((\partial_t T_t) \otimes X\right).
\]
More generally, if \(X_t\) is time dependent, then \(\partial_t\) still commutes with traces but we have the product rule for \(\otimes\):
\[
\partial_t \left(\Tr (T_t \otimes X_t)\right) = \Tr \left((\partial_t T_t) \otimes X\right) + \Tr \left(T_t \otimes (\partial_t X)\right).
\]
For better readability, we will tend to use less parenthesis in what follows which should not cause any confusion.

It should be clear to the reader that we may also trace \(T\) with a one-form \(\alpha\) and obtain similar rules, that we may trace with respect to different slots of \(T\), and that we may iterate traces as long as there are remaining slots in \(T\) and obtain similar formulae. Finally, the reader is advised to check explicitly that \(\partial_t\) commutes with traces and obeys the \(\otimes\) product rule directly from the definition of \(\partial_t T_t\).
\end{rem}

\subsection*{Splitting of The Pull-Back Bundle}

Throughout this section \(F_t : M \to N\) will be a one-parameter family of immersions. Then we have the splitting
\[
F_t^{\ast} TN \simeq TM \oplus \nu_t M
\]
with we decompose the variation vector
\[
V_t = V^{\tang}_t + V^{\perp}_t = V^{\tang}_t + \varphi_t \nu_t
\]
into the tangential \(V^{\tang}_t\) and normal \(V^{\perp}_t\) parts and where
\[
\varphi_t = \inpr{V_t}{\nu_t}.
\]

\begin{rem}
Most of our variations result in terms involving \(DV\) the covariant derviative of \(V\). Note that \((DV)^{\tang} \ne D(V^{\tang})\) in general, and likewise for the normal part. Expressed somewhat cumbersomely we have
\[
\begin{split}
DV &= D(V^{\tang} + \varphi \nu) = D(V^{\tang}) + d\varphi \otimes \nu - \varphi\W \\
&= \underbrace{(D(V^{\tang}))^{\tang} - \varphi\W}_{(DV)^{\tang}} + \underbrace{(D(V^{\tang}))^{\perp} + d\varphi \otimes \nu}_{(DV)^{\perp}}.
\end{split}
\]
In the particular case that \(V^{\tang} \equiv 0\) we have the more appealing,
\[
DV = -\varphi\W + d\varphi \otimes \nu.
\]
However, we will encounter ample situations where \(V \not\equiv 0\). To avoid the cumbersome notation above, in this section we will tend to avoid using the decomposition \(V = V^{\tang} + V^{\perp}\), deferring it to more explicit situations in later chapters as needed.
\end{rem}

\subsection*{Variation of Intrinsic Geometry}

\begin{lemma}[Variation of the Metric]
\label{lem:dt_g}
The metric \(g_t\) induced by \(F_t\) has variation
\[
\partial_t g_t = \sym \left[(DV)^{\tang}\right]^{\flat}.
\]
\end{lemma}

\begin{proof}
We have
\[
\begin{split}
\partial_t g (X, Y) &= \partial_t \left[g(X, Y)\right] = \partial_t \inpr{F_{\ast} X}{F_{\ast} Y} \\
&= \inpr{D_V F_{\ast} X}{F_{\ast} Y} + X \leftrightarrow Y.
\end{split}
\]
Recalling that we have \([V, F_{\ast} X] = 0\), then we compute
\[
\begin{split}
\inpr{D_V F_{\ast} X}{F_{\ast} Y} &= \inpr{D_{F_{\ast} X} V}{F_{\ast} Y} \\
&= \inpr{(D_{F_{\ast} X} V)^{\tang}}{F_{\ast} Y} \\
&= g((DV)^{\tang} (X), Y)
\end{split}
\]
whence
\[
\partial_t g (X, Y) = \sym g((DV)^{\tang} (X), Y).
\]
\end{proof}

From the metric \(g_t\), we may form the Levi-Civita connection \(\nabla_t\) and also compute it's variation. Since the connection is determined uniquely from the metric, we express the variation of the connection in terms of the variation of the metric which we write,
\[
\partial_t g = k
\]
for \(k\) a symmetric \((0, 2)\)-tensor. Note that since the metric is symmetric, \(k\) must be symmetric: \(k(X, Y) = \partial_t (g(X, Y)) = \partial_t (g(Y, X)) = k(Y, X)\). When the variation of \(g_t\) is induced by the variation \(V\) of \(F_t\), \Cref{lem:dt_g} gives us \(k(X, Y) = g((DV)^{\tang} (X), Y) + g(X, (DV)^{\tang} (Y))\).

\begin{rem}
Let us also note that although \(\nabla\) is not a tensor, we may still make the definition
\[
(\partial_t \nabla_t)_X Y = \partial_t((\nabla_t)_X Y)
\]
and compute the variation. To help illustrate that this is well defined, let
\[
\alpha_t(X, Y) = (\nabla_{t})_X Y - (\nabla_{0})_X Y.
\]
Because \(\nabla_t\) and \(\nabla_0\) both satisfy the Leibniz rule, \(\alpha_t\) is tensorial in both \(X\) and \(Y\) so that \(\partial_t \alpha_t\) is well defined. Then we have,
\[
(\partial_t \nabla_t)_X Y = \partial_t((\nabla_t)_X Y) = \partial_t (\alpha(X, Y) + (\nabla_{0})_X Y) = \partial_t \alpha (X, Y).
\]
Finally, observe that \(\partial_t \nabla_t = \partial_t \alpha\) is tensorial, and thus \emph{not} a connection, in contrast for example to the variation of a tensor, \(\partial_t T_t\) which is again a tensor of the same type as \(T\).
\end{rem}

We make use of the alternating, cyclic sum \(\altcyc\) defined by
\[
\altcyc T(X, Y, Z) = T(X, Y, Z) - T(Y, Z, X) + T(Z, X, Y).
\]

\begin{lemma}[Variation of the Levi-Civita Connection]
The Levi-Civita, \(\nabla_t\) induced by the metric variation \(\partial_t g = k\) has variation satisfying,
\[
g(\partial_t (\nabla_t)_X Y, Z) = \frac{1}{2} \altcyc \nabla_X k(Z, Y)
\]
If the metric variation is induced by the variation \(V\) of \(F_t\), then
\[
g(\partial_t (\nabla_t)_X Y, Z) = \frac{1}{2} \altcyc \nabla_X \sym g((DV)^{\tang} (Z), Y).
\]
\end{lemma}

\begin{proof}
We use the Koszul formula,
\[
\begin{split}
2 g(\nabla_X Y, Z) &= \partial_X g(Z, Y) - \partial_Z g(Y, X) + \partial_Y g(X, Z) \\
&\quad + g([X, Y], Z) - g([Y, Z], X) + g(Z, [X, Y]) \\
&= \altcyc \left(\partial_X g(Z, Y) + g([X, Y], Z)\right).
\end{split}
\]
Keep in mind here that everything takes place on the tangent bundle so the only \(t\) dependence is in \(g\) and \(\nabla\). In particular, the Lie bracket is simply the Lie bracket on \(M\) which has no \(t\) dependence.

Differentiating the Koszul formula and using \(\partial_t \partial_X = \partial_X \partial_t\) gives
\begin{equation}
\label{eq:dt_koszul}
\begin{split}
2 g(\partial_t \nabla_X Y, Z) &= -2 \partial_t g(\nabla_X Y, Z) + \altcyc \left(\partial_X \partial_t g(Z, Y) + \partial_t g([X, Y], Z)\right) \\
&= -2 k (\nabla_X Y, Z) + \altcyc \left(\partial_X k (Z, Y) + k ([X, Y], Z)\right).
\end{split}
\end{equation}
For the alternating, cyclic sum of the first term we have,
\begin{align*}
\partial_X k (Z, Y) &= \nabla_X k (Z, Y) + k(\nabla_X Z, Y) + k(Z, \nabla_X Y) \\
-\partial_Z k (Y, X) &= -\nabla_Z k (Y, X) - k(\nabla_Z Y, X) - k(Y, \nabla_Z X) \\
\partial_Y k (X, Z) &= \nabla_Y k (X, Z) + k(\nabla_Y X, Z) + k(X, \nabla_Y Z)
\end{align*}
Adding these together produces
\[
\begin{split}
\altcyc \partial_X k (Z, Y) &= \altcyc \nabla_X k (Z, Y) \\
&\quad + k(\nabla_X Z, Y) - k(Y, \nabla_Z X) + k(X, \nabla_Y Z) - k(\nabla_Z Y, X) \\
&\quad + k(\nabla_Y X, Z) + k(Z, \nabla_X Y) \\
&= \altcyc \nabla_X k (Z, Y) - k([Z, X], Y) + k([Y, Z], X) - k([X, Y], Z) \\
&\quad + 2 k(\nabla_X Y, Z) \\
&= \altcyc \left(\nabla_X k (Z, Y) - k ([X, Y], Z)\right) + 2 k (\nabla_X Y, Z) .
\end{split}
\]
Substitution into equation \eqref{eq:dt_koszul} then completes the proof of the first part.

For the second part, just use \Cref{lem:dt_g}.
\end{proof}

\subsection*{Variation of Extrinsic Geometry}

Next, let us compute the variation of the normal. Again, the normal lies in the changing normal bundle and we must account for that change in our variation. Note also that there are two choices for the unit normal corresponding to the two choices of orientation for \(M_t\). The variation holds regardless of our choice which the reader may verify explicitly by making the change \(\nu_t \mapsto -\nu_t\).

\begin{lemma}[Variation of the Unit Normal]
\label{lem:nu_t}
The unit normal \(\nu_t\) induced by \(F_t\) has variation satisfying
\[
g(D_t \nu_t, F_{\ast} X) = - \inpr{\nu}{(DV)^{\perp} (F_{\ast} X)}.
\]
\end{lemma}

\begin{proof}
The equations defining \(\nu_t\) are
\begin{align}
\label{eq:nu_unit}
\inpr{\nu_t}{\nu_t} &= 1. \\
\label{eq:nu_normal}
\inpr{\nu_t}{F_{\ast} X} &= 0.
\end{align}

Differentiating \eqref{eq:nu_unit} gives
\[
0 = 2 \inpr{D_t \nu_t}{\nu_t}
\]
so that \(D_t \nu_t\) is tangential.

Differentiating \eqref{eq:nu_normal} gives
\[
0 = \inpr{D_t \nu_t}{F_{\ast} X} + \inpr{\nu_t}{D_{F_{\ast} X} V}
\]
so that
\[
\inpr{D_t \nu}{F_{\ast} X} = - \inpr{\nu_t}{DV(F_{\ast} X)} = - \inpr{\nu_t}{(DV)^{\perp}(F_{\ast} X)}
\]
\end{proof}
